\documentclass{article}
\usepackage[utf8]{inputenc}
\usepackage{graphicx,amsmath,amsfonts,amssymb}
\usepackage{enumitem}
\usepackage{xcolor}

\newcommand\Ra{\mathrm{Ra}}
\newcommand\Pran{\mathrm{Pr}}
\newcommand\Rac{\mathrm{Ra}_{\mathrm{c}}}
\newcommand\Ek{\mathrm{Ek}}
\newcommand\Ro{\mathrm{Ro}}
\newcommand\Nu{\mathrm{Nu}}
\newcommand\Sc{\mathrm{Sc}}

\newcommand\eps{\varepsilon}
\renewcommand\L {\mathcal{L}}

\newcommand{\n}{\\ \nonumber \\ }
\newcommand{\nn}{\nonumber}
\newcommand{\nnn}{\\ \nonumber \\ \nonumber}

\newcommand\ie{\textit{i.e.},~}
\newcommand\eg{\textit{e.g.},~}
\newcommand{\omicron}{o}

\newcommand{\pd}[1]{\partial_{#1}}
\newcommand{\vhat}[1]{\hat{\boldsymbol{#1}}}
\renewcommand{\vec}[1]{\boldsymbol{#1}}
\newcommand{\M}[1]{\mathbf{#1}}
\newcommand{\grad}{\vec{\nabla}}
\newcommand{\cross}{\vec{\times}}
\newcommand{\curl}{\grad \vec{\times}}
\newcommand{\laplacian}{\nabla^2}
\newcommand{\veclaplacian}{\grad^2}

\newcommand{\sump}[2]{\sideset{}{'}\sum_{{#1}=0}^{#2}}

\newcommand{\eq}[1]{(\ref{#1})}
\newcommand{\eqs}[2]{(\ref{#1})~\&~(\ref{#2})}
\newcommand{\eqss}[2]{(\ref{#1})--(\ref{#2})}

\newcommand{\Eq}[1]{Eq.~(\ref{#1})}
\newcommand{\Eqs}[2]{Eqs.~(\ref{#1})~\&~(\ref{#2})}
\newcommand{\Eqss}[2]{Eqs.~(\ref{#1})--(\ref{#2})}

\newcommand{\fig}[1]{Fig.~(\ref{#1})}
\newcommand{\figs}[2]{Figs.~(\ref{#1})~\&~(\ref{#2})}
\newcommand{\T}{{\cal T}}
\newcommand{\Z}{{\cal Z}}

\setlength{\parindent}{0pt}

\begin{document}

\section{MRI Notes}
\subsection{Daniel (8/6/2021)}
\subsubsection{Potential Form}
The induction equation is given by
\begin{align*}
    \partial_t\vec{b} &= \grad \cross (\vec{u} \cross \vec{b}) + \eta \grad^2 \vec{b}
    \intertext{Using the following identity}
    \grad \cross \grad \cross \vec{f} &= \grad \grad \cdot \vec{f} - \grad^2\vec{f}
    \intertext{and assumming $\eta$ to be constant, we use $\grad \cdot \vec{b} = 0$, giving}
    \partial_t\vec{b} &= \grad \cross (\vec{u} \cross \vec{b}) - \eta \grad \cross \grad \cross \vec{b}.
    \intertext{Then we define a vector potential $\grad \cross \vec{A} \equiv \vec{b}$, yielding}
    \partial_t \grad \cross \vec{A} &= \grad \cross (\vec{u} \cross \vec{b}) - \grad \cross (\eta \grad \cross \vec{b}) \\
    \partial_t \vec{A} &= \vec{u} \cross \vec{b} - \eta \grad \cross \vec{b} + \grad \phi
    \intertext{where $\phi$ is a scalar potential arizing from ``uncurling'' the equation. We must then provide an additional constraint to fix $\phi$: the Coulomb gauge $\grad \cdot \vec{A} = 0$. Therefore}
    -\grad \cross \vec{b} &= - \curl \grad \cross \vec{A} = \veclaplacian \vec{A}.
\end{align*}
Next we decompose $\vec{u} = \vec{u_0} + \vec{u'}$ and $\vec{b} = \vec{b_0} + \vec{b'}$.
We assume the mean quantities $\vec{u_0}$ and $\vec{b_0}$ are themselves solutions to the original problem. 
If we consider only the 0th mode of $\vec{b}$, i.e. $\vec{b} \cdot \vec{\hat{e}_i} \sim e^{i0}$ then clearly $\veclaplacian \vec{b} = \vec{0}$ and therefore
\begin{align*}
    \partial_t \vec{b'} &= \curl (\vec{u_0} \cross \vec{b'}) + \curl (\vec{u'} \cross \vec{b_0}) + \curl (\vec{u'} \cross \vec{b'}).
    \intertext{Using another identity}
    \curl (\vec{A} \cross \vec{B}) &= \vec{A} \grad \cdot \vec{B} - \vec{B} \grad \cdot \vec{A} + \vec{B} \cdot \grad \vec{A} - \vec{A} \cdot \grad \vec{B}
\end{align*}

\subsection*{Momentum Equation}
Verbatim from Jeff Oishi's MRI paper:
\begin{align*}
    \frac{D\vec{u'}}{Dt} + f\vhat{z} \cross\vec{u'} + Su_x' \vhat{y} + \grad p' + \nu \curl \omega' &= B_0 \partial_z \vec{b'}
    \intertext{where $f$ is the corriolis parameter, S is the background shearing rate, and $B_0 \vhat{z}$ is a uniform background magnetic field. This equation is linearized wrt perturbations. Accordingly, the material derivative goes like}
    \frac{D}{Dt} &\equiv \partial_t + \vec{u} \cdot \grad \\
    &= \partial_t + Sx\partial_y
    \intertext{due to the background velocity $\vec{\overline{u}} = Sx\vhat{y}$. 
    In the nonlinear case we have}
    &= \partial_t + (Sx\vhat{y} + \vec{u'}) \cdot \grad
    \intertext{From inspection and stuff, the irrotational momentum equation goes like}
    \frac{D\vec{u}}{Dt} + \grad p + \nu\cross \omega &= \vec{b} \cdot \grad\vec{b}
    \intertext{Next we generalize $\vec{u} = \vec{u'} + Sx\vhat{y}$ and $\vec{b} = \vec{b'} + B_0\vhat{z}$, giving}
    \underline{\partial_t \vec{u'} + \vec{u'} \cdot \grad \vec{u'} + Sx\partial_y \vec{u'} + Su'_x \vhat{y}} + \grad p + \nu \curl \omega &= B_0 \partial_z\vec{b'} + \vec{b'} \cdot \grad \vec{b'}
    \intertext{where the material derivative $\frac{D\vec{u}}{Dt}$ consists of the underlined terms}
\end{align*}

\end{document}
