\documentclass{article}
\usepackage[utf8]{inputenc}
\usepackage{graphicx,amsmath,amsfonts,amssymb,mathtools}
\usepackage{enumitem}
\usepackage{empheq}
\usepackage{xcolor}

\newcommand\Ra{\mathrm{Ra}}
\newcommand\Pran{\mathrm{Pr}}
\newcommand\Rac{\mathrm{Ra}_{\mathrm{c}}}
\newcommand\Ek{\mathrm{Ek}}
\newcommand\Ro{\mathrm{Ro}}
\newcommand\Nu{\mathrm{Nu}}
\newcommand\Sc{\mathrm{Sc}}

\newcommand\eps{\varepsilon}
\renewcommand\L {\mathcal{L}}

\newcommand{\n}{\\ \nonumber \\ }
\newcommand{\nn}{\nonumber}
\newcommand{\nnn}{\\ \nonumber \\ \nonumber}

\newcommand\ie{\textit{i.e.},~}
\newcommand\eg{\textit{e.g.},~}
\newcommand{\omicron}{o}
\newcommand*\widefbox[1]{\fbox{\hspace{2em}#1\hspace{2em}}}

\newcommand{\pd}[1]{\partial_{#1}}
\newcommand{\vhat}[1]{\hat{\boldsymbol{#1}}}
\renewcommand{\vec}[1]{\boldsymbol{#1}}
\newcommand{\M}[1]{\mathbf{#1}}
\newcommand{\grad}{\vec{\nabla}}
\newcommand{\cross}{\vec{\times}}
\newcommand{\curl}{\grad \vec{\times}}
\newcommand{\divergence}{\grad \cdot}
\newcommand{\laplacian}{\nabla^2}
\newcommand{\veclaplacian}{\grad^2}

\newcommand{\sump}[2]{\sideset{}{'}\sum_{{#1}=0}^{#2}}

\newcommand{\eq}[1]{(\ref{#1})}
\newcommand{\eqs}[2]{(\ref{#1})~\&~(\ref{#2})}
\newcommand{\eqss}[2]{(\ref{#1})--(\ref{#2})}

\newcommand{\Eq}[1]{Eq.~(\ref{#1})}
\newcommand{\Eqs}[2]{Eqs.~(\ref{#1})~\&~(\ref{#2})}
\newcommand{\Eqss}[2]{Eqs.~(\ref{#1})--(\ref{#2})}

\newcommand{\fig}[1]{Fig.~(\ref{#1})}
\newcommand{\figs}[2]{Figs.~(\ref{#1})~\&~(\ref{#2})}
\newcommand{\T}{{\cal T}}
\newcommand{\Z}{{\cal Z}}

\setlength{\parindent}{0pt}

\begin{document}
\subsection*{Linear Partial Differential Equations}
\vspace{1cm}
\begin{enumerate}
    \item Consider the \textbf{diffusion equation}
    \begin{align*}
        \partial_t u &= a \partial_x^2 u \tag{1}
        \intertext{on the domain $x \in [0, 2\pi).$}
        \intertext{Derive the solution to this equation, assumming the initial state is given by an arbitrary Fourier mode, i.e.}
        u(x, 0) &= \sin(nx) \tag{2}
        \intertext{where $n$ is a natural number ($n \in \mathbb{N}$)}.
        \intertext{\textbf{Hint: } assume the solution is of the form}
        u(x, t) &= \sin(nx) f(t), \tag{3}
        \intertext{then substitute (3) into (1) and solve for $f(t)$. The initial condition (2) places a constraint on $f(0)$, so your solution should be unique (there shouldn't be any undetermined constants or coefficients).}
    \end{align*}

    If you can get this solution for a single arbitrary Fourier mode, then all you have to do for a more general initial state $u(x, 0)$ is find its Fourier coefficients and treat them each individually. This is the magic of linearity!

    \begin{align*}
        \intertext{Solutions go like}
        u(x, t) &= e^{-an^2t} \sin(nx)
    \end{align*}

\end{enumerate}

\end{document}
 