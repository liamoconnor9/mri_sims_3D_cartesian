\documentclass{article}
\usepackage[utf8]{inputenc}
\usepackage{graphicx,amsmath,amsfonts,amssymb,mathtools}
\usepackage{enumitem}
\usepackage{empheq}
\usepackage{xcolor}

\newcommand\Ra{\mathrm{Ra}}
\newcommand\Pran{\mathrm{Pr}}
\newcommand\Rac{\mathrm{Ra}_{\mathrm{c}}}
\newcommand\Ek{\mathrm{Ek}}
\newcommand\Ro{\mathrm{Ro}}
\newcommand\Nu{\mathrm{Nu}}
\newcommand\Sc{\mathrm{Sc}}

\newcommand\eps{\varepsilon}
\renewcommand\L {\mathcal{L}}

\newcommand{\n}{\\ \nonumber \\ }
\newcommand{\nn}{\nonumber}
\newcommand{\nnn}{\\ \nonumber \\ \nonumber}

\newcommand\ie{\textit{i.e.},~}
\newcommand\eg{\textit{e.g.},~}
\newcommand{\omicron}{o}
\newcommand*\widefbox[1]{\fbox{\hspace{2em}#1\hspace{2em}}}

\newcommand{\pd}[1]{\partial_{#1}}
\newcommand{\vhat}[1]{\hat{\boldsymbol{#1}}}
\renewcommand{\vec}[1]{\boldsymbol{#1}}
\newcommand{\M}[1]{\mathbf{#1}}
\newcommand{\grad}{\vec{\nabla}}
\newcommand{\cross}{\vec{\times}}
\newcommand{\curl}{\grad \vec{\times}}
\newcommand{\divergence}{\grad \cdot}
\newcommand{\laplacian}{\nabla^2}
\newcommand{\veclaplacian}{\grad^2}

\newcommand{\sump}[2]{\sideset{}{'}\sum_{{#1}=0}^{#2}}

\newcommand{\eq}[1]{(\ref{#1})}
\newcommand{\eqs}[2]{(\ref{#1})~\&~(\ref{#2})}
\newcommand{\eqss}[2]{(\ref{#1})--(\ref{#2})}

\newcommand{\Eq}[1]{Eq.~(\ref{#1})}
\newcommand{\Eqs}[2]{Eqs.~(\ref{#1})~\&~(\ref{#2})}
\newcommand{\Eqss}[2]{Eqs.~(\ref{#1})--(\ref{#2})}

\newcommand{\fig}[1]{Fig.~(\ref{#1})}
\newcommand{\figs}[2]{Figs.~(\ref{#1})~\&~(\ref{#2})}
\newcommand{\T}{{\cal T}}
\newcommand{\Z}{{\cal Z}}

\setlength{\parindent}{0pt}

\begin{document}

\section*{MRI Notes}
\subsection*{Induction Equation: Potential Form}
The induction equation is given by
\begin{align*}
    \partial_t\vec{b} &= \grad \cross (\vec{u} \cross \vec{b}) + \eta \grad^2 \vec{b}
    \intertext{Using the following identity}
    \grad \cross \grad \cross \vec{f} &= \grad \grad \cdot \vec{f} - \grad^2\vec{f}
    \intertext{and assumming $\eta$ to be constant, we use $\grad \cdot \vec{b} = 0$, giving}
    \partial_t\vec{b} &= \grad \cross (\vec{u} \cross \vec{b}) - \eta \grad \cross \grad \cross \vec{b}.
    \intertext{Then we define a vector potential $\grad \cross \vec{A} \equiv \vec{b}$, yielding}
    \partial_t \grad \cross \vec{A} &= \grad \cross (\vec{u} \cross \vec{b}) - \grad \cross (\eta \grad \cross \vec{b}) \\
    \partial_t \vec{A} &= \vec{u} \cross \vec{b} - \eta \grad \cross \vec{b} + \grad \phi
    \intertext{where $\phi$ is a scalar potential arizing from ``uncurling'' the equation. We must then provide an additional constraint to fix $\phi$: the Coulomb gauge $\grad \cdot \vec{A} = 0$. Therefore}
    -\grad \cross \vec{b} &= - \curl \grad \cross \vec{A} = \veclaplacian \vec{A}. \\
    \partial_t \vec{A} &= \vec{u} \cross (\grad \cross \vec{A}) - \eta \veclaplacian \vec{A} + \grad \phi
\end{align*}
Next we decompose $\vec{u} = \vec{u_0} + \vec{u'}$ and $\vec{b} = \vec{b_0} + \vec{b'} \; \to \; \vec{A} = \vec{A_0} + \vec{A'}$.
We assume the mean quantities $\vec{u_0}$ and $\vec{A_0}$ are themselves solutions to the original problem. 
If we consider only the 0th mode of $\vec{b}$, i.e. $\vec{b} \cdot \vec{\hat{e}_i} \sim e^{i0}$ then clearly $\veclaplacian \vec{b} = \vec{0}$ and therefore
\begin{align*}
    \partial_t \vec{b'} &= \curl (\vec{u_0} \cross \vec{b'}) + \curl (\vec{u'} \cross \vec{b_0}) + \curl (\vec{u'} \cross \vec{b'}).
    \intertext{Using another identity}
    \curl (\vec{A} \cross \vec{B}) &= \vec{A} \grad \cdot \vec{B} - \vec{B} \grad \cdot \vec{A} + \vec{B} \cdot \grad \vec{A} - \vec{A} \cdot \grad \vec{B}
\end{align*}

\pagebreak
\begin{align*}
    \intertext{Let's check that. Using the following identity}
    \curl\curl (\vec{u} \cross\vec{b}) &= \curl\big(\vec{u}(\divergence \vec{b}) - \vec{b}(\divergence\vec{u}) + (\vec{b}\cdot\grad)\vec{u} - (\vec{u}\cdot\grad)\vec{b}\big) \\
    &= \curl\big((\vec{b}\cdot\grad)\vec{u} - (\vec{u}\cdot\grad)\vec{b}\big) \\
    &= \curl\Bigg(\begin{bmatrix} \vec{b}\cdot\grad u_x\\\vec{b}\cdot\grad u_y\\\vec{b}\cdot\grad u_z\end{bmatrix} - \begin{bmatrix} \vec{u}\cdot\grad b_x\\\vec{u}\cdot\grad b_y\\\vec{u}\cdot\grad b_z \end{bmatrix}\Bigg) \\
    \intertext{Taking the $x$-component gives}
    \vhat{x}\cdot\curl\curl (\vec{u} \cross\vec{b}) &= \partial_y(\vec{b}\cdot\grad u_z - \vec{u}\cdot\grad b_z) - \partial_z(\vec{b}\cdot\grad u_y - \vec{u}\cdot\grad b_y) \\
    &= \partial_y\vec{b}\cdot\grad u_z + \vec{b}\cdot\grad\partial_y u_z - \partial_y\vec{u}\cdot\grad b_z - \vec{u}\cdot\grad\partial_y b_z \\
    &\qquad - \partial_z\vec{b}\cdot\grad u_y - \vec{b}\cdot\grad\partial_z u_y + \partial_z\vec{u}\cdot\grad b_y + \vec{u}\cdot\grad\partial_z b_y \\
    &= \vec{b}\cdot\grad\omega_x - \vec{u}\cdot\grad j_x \\
    &\qquad + \partial_y\vec{b}\cdot\grad u_z - \partial_z\vec{b}\cdot\grad u_y + \partial_z\vec{u}\cdot\grad b_y - \partial_y\vec{u}\cdot\grad b_z \\
    \intertext{Letting $B_0=S=0$, the expanded current-density equation is given by}
    \partial_tj_x &= \eta\laplacian j_x \\
    &\qquad + \vec{b} \cdot\grad\omega_x - \vec{u}\cdot\grad j_x\\
    &\qquad +  \partial_y\vec{b}\cdot\grad u_z - \partial_z \vec{b}\cdot\grad u_y + \partial_z\vec{u}\cdot\grad b_y - \partial_y\vec{u}\cdot\grad b_z \\
    &= \eta\laplacian j_x + \vhat{x}\cdot \curl\curl (\vec{u} \cross \vec{b}).
    \intertext{by symmetry we have}
    \partial_t \vec{j} &= \eta\veclaplacian \vec{j} + \curl\curl (\vec{u} \cross \vec{b})
    \intertext{We can ``uncurl'' this equation, and if we assume the unknown scalar potential-gradient term $\grad \phi = \vec{0}$, we have}
    \partial_t \vec{b} &= \curl (\vec{u}\cross\vec{b}) + \eta\veclaplacian\vec{b}
    \intertext{which is the usual form of the induction equation}
\end{align*}
\newpage

\begin{align*}
    \intertext{Navier-Stokes:}
    &\partial_t \vec{u} + Sx\partial_y \vec{u} + f\hat{z}\times\vec{u} + [S\vec{u}\cdot\hat{x}]\hat{y} + \grad p - \nu \laplacian \vec{u} - \vec{B}\cdot \grad\vec{b} = \vec{b} \cdot \grad\vec{b} - \vec{u} \cdot \grad\vec{u}\\
    \intertext{Induction:}
    &\partial_t \vec{A} -\grad\phi -\eta\laplacian\vec{A} - \vec{u}\times\vec{B} - Sx\hat{y}\times\vec{b} = \vec{u}\times\vec{b} \\
    \intertext{Incompressibility and Coulomb gauge}
    &\grad\cdot\vec{u} = \grad \cdot\vec{A} = 0\\
    \intertext{$\vec{A}$ definition}
    &\vec{b} = \curl \vec{A} 
    \intertext{We define the Lagrangian with multipliers $\vec{\mu},\vec{\Lambda},\pi,\alpha,\vec{\beta}$}
    &\mathcal{L} \equiv \int_0^T \Big\langle \vec{\mu}\cdot\Big[ \partial_t \vec{u} + Sx\partial_y \vec{u} + f\hat{z}\times\vec{u} + [S\vec{u}\cdot\hat{x}]\hat{y} + \grad p - \nu \laplacian \vec{u} - \vec{B}\cdot \grad\vec{b} - \vec{b} \cdot \grad\vec{b} + \vec{u} \cdot \grad\vec{u} \Big] \Big\rangle dt \\
    &+ \int_0^T \Big\langle\vec{\Lambda}\cdot\Big[ \partial_t \vec{A} -\grad\phi -\eta\laplacian\vec{A} - \vec{u}\times\vec{B} - Sx\hat{y}\times\vec{b} - \vec{u}\times\vec{b} \Big]  \Big\rangle dt \\
    &+ \int_0^T \Big\langle \pi \grad\cdot\vec{u} \Big\rangle dt + \int_0^T \Big\langle \alpha \grad\cdot\vec{A} \Big\rangle dt + \int_0^T \Big\langle \vec{\beta} \cdot \Big[ \vec{b} - \curl\vec{A} \Big]  \Big\rangle dt
    \intertext{The variation with respect to $\vec{u}$}
    &\frac{\delta \mathcal{L}}{\delta \vec{u}}\delta \vec{u} = \\
    &\int_0^T \Big\langle \delta\vec{u} \cdot \Big[ -\partial_t\mu - Sx\partial_y\vec{\mu} + f\hat{z}\times\vec{\mu} + [S\vec{\mu}\cdot\hat{y}]\hat{x} + (\grad\vec{u})^T \cdot \vec{\mu} - \vec{u}\cdot\vec{\mu} - \nu\laplacian\vec{\mu} - \vec{\Lambda}\times\vec{B} - \vec{\Lambda}\times\vec{b} - \grad\pi \Big]  \Big\rangle dt
\end{align*}
\end{document}
 