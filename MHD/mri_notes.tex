\documentclass{article}
\usepackage[utf8]{inputenc}
\usepackage{graphicx,amsmath,amsfonts,amssymb,mathtools}
\usepackage{enumitem}
\usepackage{empheq}
\usepackage{xcolor}

\newcommand\Ra{\mathrm{Ra}}
\newcommand\Pran{\mathrm{Pr}}
\newcommand\Rac{\mathrm{Ra}_{\mathrm{c}}}
\newcommand\Ek{\mathrm{Ek}}
\newcommand\Ro{\mathrm{Ro}}
\newcommand\Nu{\mathrm{Nu}}
\newcommand\Sc{\mathrm{Sc}}

\newcommand\eps{\varepsilon}
\renewcommand\L {\mathcal{L}}

\newcommand{\n}{\\ \nonumber \\ }
\newcommand{\nn}{\nonumber}
\newcommand{\nnn}{\\ \nonumber \\ \nonumber}

\newcommand\ie{\textit{i.e.},~}
\newcommand\eg{\textit{e.g.},~}
\newcommand{\omicron}{o}
\newcommand*\widefbox[1]{\fbox{\hspace{2em}#1\hspace{2em}}}

\newcommand{\pd}[1]{\partial_{#1}}
\newcommand{\vhat}[1]{\hat{\boldsymbol{#1}}}
\renewcommand{\vec}[1]{\boldsymbol{#1}}
\newcommand{\M}[1]{\mathbf{#1}}
\newcommand{\grad}{\vec{\nabla}}
\newcommand{\cross}{\vec{\times}}
\newcommand{\curl}{\grad \vec{\times}}
\newcommand{\divergence}{\grad \cdot}
\newcommand{\laplacian}{\nabla^2}
\newcommand{\veclaplacian}{\grad^2}

\newcommand{\sump}[2]{\sideset{}{'}\sum_{{#1}=0}^{#2}}

\newcommand{\eq}[1]{(\ref{#1})}
\newcommand{\eqs}[2]{(\ref{#1})~\&~(\ref{#2})}
\newcommand{\eqss}[2]{(\ref{#1})--(\ref{#2})}

\newcommand{\Eq}[1]{Eq.~(\ref{#1})}
\newcommand{\Eqs}[2]{Eqs.~(\ref{#1})~\&~(\ref{#2})}
\newcommand{\Eqss}[2]{Eqs.~(\ref{#1})--(\ref{#2})}

\newcommand{\fig}[1]{Fig.~(\ref{#1})}
\newcommand{\figs}[2]{Figs.~(\ref{#1})~\&~(\ref{#2})}
\newcommand{\T}{{\cal T}}
\newcommand{\Z}{{\cal Z}}

\setlength{\parindent}{0pt}

\begin{document}

\section*{MRI Notes}
\subsection*{Induction Equation: Potential Form}
The induction equation is given by
\begin{align*}
    \partial_t\vec{b} &= \grad \cross (\vec{u} \cross \vec{b}) + \eta \grad^2 \vec{b}
    \intertext{Using the following identity}
    \grad \cross \grad \cross \vec{f} &= \grad \grad \cdot \vec{f} - \grad^2\vec{f}
    \intertext{and assumming $\eta$ to be constant, we use $\grad \cdot \vec{b} = 0$, giving}
    \partial_t\vec{b} &= \grad \cross (\vec{u} \cross \vec{b}) - \eta \grad \cross \grad \cross \vec{b}.
    \intertext{Then we define a vector potential $\grad \cross \vec{A} \equiv \vec{b}$, yielding}
    \partial_t \grad \cross \vec{A} &= \grad \cross (\vec{u} \cross \vec{b}) - \grad \cross (\eta \grad \cross \vec{b}) \\
    \partial_t \vec{A} &= \vec{u} \cross \vec{b} - \eta \grad \cross \vec{b} + \grad \phi
    \intertext{where $\phi$ is a scalar potential arizing from ``uncurling'' the equation. We must then provide an additional constraint to fix $\phi$: the Coulomb gauge $\grad \cdot \vec{A} = 0$. Therefore}
    -\grad \cross \vec{b} &= - \curl \grad \cross \vec{A} = \veclaplacian \vec{A}. \\
    \partial_t \vec{A} &= \vec{u} \cross (\grad \cross \vec{A}) - \eta \veclaplacian \vec{A} + \grad \phi
\end{align*}
Next we decompose $\vec{u} = \vec{u_0} + \vec{u'}$ and $\vec{b} = \vec{b_0} + \vec{b'} \; \to \; \vec{A} = \vec{A_0} + \vec{A'}$.
We assume the mean quantities $\vec{u_0}$ and $\vec{A_0}$ are themselves solutions to the original problem. 
If we consider only the 0th mode of $\vec{b}$, i.e. $\vec{b} \cdot \vec{\hat{e}_i} \sim e^{i0}$ then clearly $\veclaplacian \vec{b} = \vec{0}$ and therefore
\begin{align*}
    \partial_t \vec{b'} &= \curl (\vec{u_0} \cross \vec{b'}) + \curl (\vec{u'} \cross \vec{b_0}) + \curl (\vec{u'} \cross \vec{b'}).
    \intertext{Using another identity}
    \curl (\vec{A} \cross \vec{B}) &= \vec{A} \grad \cdot \vec{B} - \vec{B} \grad \cdot \vec{A} + \vec{B} \cdot \grad \vec{A} - \vec{A} \cdot \grad \vec{B}
\end{align*}

% \subsection*{Momentum Equation: Nonlinear Terms}
% Navier-Stokes (Verbatim from Jeff Oishi's ``MRI prefers'' paper):
% \begin{align*}
%     \frac{D\vec{u'}}{Dt} + f\vhat{z} \cross\vec{u'} + Su_x' \vhat{y} + \grad p' + \nu \curl \vec{\omega}' &= B_0 \partial_z \vec{b'}
%     \intertext{where $f$ is the corriolis parameter, S is the background shearing rate, and $B_0 \vhat{z}$ is a uniform background magnetic field. The equation is linearized wrt perturbations so the material derivative goes like}
%     \frac{D}{Dt} &\equiv \partial_t + \vec{\overline{u}} \cdot \grad \\
%     &= \partial_t + Sx\partial_y
%     \intertext{due to the background velocity $\vec{\overline{u}} = Sx\vhat{y}$. 
%     In the nonlinear case we have}
%     &= \partial_t + (Sx\vhat{y} + \vec{u'}) \cdot \grad
%     \intertext{From inspection and stuff, the irrotational momentum equation goes like}
%     \frac{D\vec{u}}{Dt} + \grad p + \nu\cross \vec{\omega} &= \vec{b} \cdot \grad\vec{b}
%     \intertext{Next we generalize $\vec{u} = \vec{u'} + Sx\vhat{y}$ and $\vec{b} = \vec{b'} + B_0\vhat{z}$, giving}
%     \underline{\partial_t \vec{u'} + \vec{u'} \cdot \grad \vec{u'} + Sx\partial_y \vec{u'} + Su'_x \vhat{y}} + \grad p + \nu \curl \vec{\omega} &= B_0 \partial_z\vec{b'} + \vec{b'} \cdot \grad \vec{b'}
%     \intertext{where the material derivative $\frac{D\vec{u}}{Dt}$ consists of the underlined terms. Note this definition differs from that of the associated script}
% \end{align*}

% \subsection*{Induction Equation: Nonlinear Terms}
% The MHD induction equation (\textit{Fluid Mechanics of Planets and Stars, 2019}) is given by
% \begin{align*}
%     \partial_t \vec{b} &= \underline{\curl (\vec{u} \cross \vec{b})} + \eta \laplacian \vec{b}.
%     \intertext{We expand the underlined term using the following identity}
%     \curl(\vec{u}\cross\vec{b}) &= \vec{u}\divergence\vec{b} - \vec{b}\divergence\vec{u} + \vec{b}\cdot\grad\vec{u} - \vec{u}\cdot\grad\vec{b}
%     \intertext{where the first two terms on the RHS vanish due to incompressibility and the abscence of magnetic monopoles. 
%     Therefore}
%     \partial_t \vec{b} + \vec{u} \cdot \grad\vec{b} &= \vec{b}\cdot\grad\vec{u} + \eta\laplacian\vec{b}.
%     \intertext{Substituting the decompositions for $\vec{u}$ and $\vec{b}$ as above yields}
%     \partial_t (B_0\vhat{z} + \vec{b'}) + (Sx\vhat{y} + \vec{u'}) \cdot \grad(B_0\vhat{z} + \vec{b'}) &= (B_0\vhat{z} + \vec{b'})\cdot\grad(Sx\vhat{y} + \vec{u'}) + \eta\laplacian(B_0\vhat{z} + \vec{b'}) \\
%     \partial_t \vec{b'} + (Sx\vhat{y} + \vec{u'}) \cdot \grad \vec{b'} &= (B_0\vhat{z} + \vec{b'})\cdot\grad(Sx\vhat{y} + \vec{u'}) + \eta\laplacian\vec{b'} 
%     \intertext{dropping the $'$s and taking the $x,y,$ and $z$ components of the above yields}
%     \partial_t b_x + Sx \partial_y b_x + \vec{u} \cdot \grad b_x &= B_0\partial_z u_x + \vec{b} \cdot \grad u_x + \eta\laplacian b_x \\
%     \partial_t b_y + Sx \partial_y b_y + \vec{u} \cdot \grad b_y &= B_0\partial_z u_y + \vec{b} \cdot \grad u_y + S b_x + \eta\laplacian b_y \\
%     \partial_t b_z + Sx \partial_y b_z + \vec{u} \cdot \grad b_z &= B_0\partial_z u_z + \vec{b} \cdot \grad u_z + \eta\laplacian b_z 
%     \intertext{Note that the nonlinear operator on the LHS is identical for each scalar equation. Accordingly, we redefine $D_t$ to be this operator, i.e.}
%     D_tA &\equiv \partial_t A + Sx \partial_y A + \vec{u} \cdot \grad A \\
%     \intertext{\textbf{NOTE: the material derivative substitution implemented in \texttt{Dedalus} excludes the nonlinear term.}}
%     \intertext{We continue by taking $\partial_z$ of the $b_y$ equation}
%     \partial_z D_t b_y = \partial_z \Big[ \partial_t b_y + Sx \partial_y b_y + \vec{u} \cdot \grad b_y \Big] &= \partial_z  B_0\partial_z u_y + \partial_z [\vec{b} \cdot \grad u_y] + S \partial_z b_x + \eta\laplacian \partial_z b_y \\
%     D_t \partial_z b_y + \partial_z \vec{u} \cdot \grad b_y &= B_0\partial^2_z u_y + \partial_z \vec{b} \cdot \grad u_y + \vec{b} \cdot \grad\partial_z u_y + S \partial_z b_x + \eta\laplacian \partial_z b_y \\
%     \intertext{and the $\partial_y$ of the $b_z$ equation}
%     \partial_y D_t b_z = \partial_y \Big[ \partial_t b_z + Sx \partial_y b_z + \vec{u} \cdot \grad b_z \Big] &= B_0\partial_z \partial_yu_z + \partial_y[\vec{b} \cdot \grad u_z] + \eta\laplacian \partial_yb_z \\
%     D_t \partial_y b_z +\partial_y \vec{u}\cdot\grad b_z &= B_0\partial_z \partial_yu_z + \partial_y\vec{b} \cdot \grad u_z + \vec{b} \cdot \grad \partial_yu_z + \eta\laplacian \partial_yb_z 
%     \intertext{Recall that the current density is given by}
%     \vec{j} = j_x\vhat{x} + j_y\vhat{y} + j_z\vhat{z} &= \curl \vec{b} \\
%     j_x &= \partial_y b_z - \partial_z b_y \\
%     D_t j_x &= D_t\partial_y b_z - D_t\partial_z b_y 
%     \intertext{Therefore}
%     D_t j_x + \partial_y\vec{u}\cdot\grad b_z - \partial_z\vec{u}\cdot\grad b_y = B_0\partial_z \omega_x + \vec{b} \cdot\grad\omega_x - &S\partial_zb_x + \eta\laplacian j_x + \partial_y\vec{b}\cdot\grad u_z - \partial_z \vec{b}\cdot\grad u_y
%     \intertext{Expanding the material derivative and grouping linear terms on the LHS gives}
% \end{align*}
% \begin{subequations}
% \begin{empheq}[box=\widefbox]{align*}
%     \partial_t j_x + Sx \partial_y j_x - B_0\partial_z \omega_x + S\partial_zb_x - \eta\laplacian j_x &=
%     \vec{b} \cdot\grad\omega_x - \vec{u} \cdot \grad j_x\nonumber\\&\quad + \partial_y\vec{b}\cdot\grad u_z - \partial_z \vec{b}\cdot\grad u_y \\\nonumber&\quad- \partial_y\vec{u}\cdot\grad b_z+ \partial_z\vec{u}\cdot\grad b_y\nonumber
% \end{empheq}
% \end{subequations}

% \pagebreak
% \begin{align*}
%     \intertext{Let's check that. Using the following identity}
%     \curl\curl (\vec{u} \cross\vec{b}) &= \curl\big(\vec{u}(\divergence \vec{b}) - \vec{b}(\divergence\vec{u}) + (\vec{b}\cdot\grad)\vec{u} - (\vec{u}\cdot\grad)\vec{b}\big) \\
%     &= \curl\big((\vec{b}\cdot\grad)\vec{u} - (\vec{u}\cdot\grad)\vec{b}\big) \\
%     &= \curl\Bigg(\begin{bmatrix} \vec{b}\cdot\grad u_x\\\vec{b}\cdot\grad u_y\\\vec{b}\cdot\grad u_z\end{bmatrix} - \begin{bmatrix} \vec{u}\cdot\grad b_x\\\vec{u}\cdot\grad b_y\\\vec{u}\cdot\grad b_z \end{bmatrix}\Bigg) \\
%     \intertext{Taking the $x$-component gives}
%     \vhat{x}\cdot\curl\curl (\vec{u} \cross\vec{b}) &= \partial_y(\vec{b}\cdot\grad u_z - \vec{u}\cdot\grad b_z) - \partial_z(\vec{b}\cdot\grad u_y - \vec{u}\cdot\grad b_y) \\
%     &= \partial_y\vec{b}\cdot\grad u_z + \vec{b}\cdot\grad\partial_y u_z - \partial_y\vec{u}\cdot\grad b_z - \vec{u}\cdot\grad\partial_y b_z \\
%     &\qquad - \partial_z\vec{b}\cdot\grad u_y - \vec{b}\cdot\grad\partial_z u_y + \partial_z\vec{u}\cdot\grad b_y + \vec{u}\cdot\grad\partial_z b_y \\
%     &= \vec{b}\cdot\grad\omega_x - \vec{u}\cdot\grad j_x \\
%     &\qquad + \partial_y\vec{b}\cdot\grad u_z - \partial_z\vec{b}\cdot\grad u_y + \partial_z\vec{u}\cdot\grad b_y - \partial_y\vec{u}\cdot\grad b_z \\
%     \intertext{Letting $B_0=S=0$, the expanded current-density equation is given by}
%     \partial_tj_x &= \eta\laplacian j_x \\
%     &\qquad + \vec{b} \cdot\grad\omega_x - \vec{u}\cdot\grad j_x\\
%     &\qquad +  \partial_y\vec{b}\cdot\grad u_z - \partial_z \vec{b}\cdot\grad u_y + \partial_z\vec{u}\cdot\grad b_y - \partial_y\vec{u}\cdot\grad b_z \\
%     &= \eta\laplacian j_x + \vhat{x}\cdot \curl\curl (\vec{u} \cross \vec{b}).
%     \intertext{by symmetry we have}
%     \partial_t \vec{j} &= \eta\veclaplacian \vec{j} + \curl\curl (\vec{u} \cross \vec{b})
%     \intertext{We can ``uncurl'' this equation, and if we assume the unknown scalar potential-gradient term $\grad \phi = \vec{0}$, we have}
%     \partial_t \vec{b} &= \curl (\vec{u}\cross\vec{b}) + \eta\veclaplacian\vec{b}
%     \intertext{which is the usual form of the induction equation}
% \end{align*}
\end{document}
 