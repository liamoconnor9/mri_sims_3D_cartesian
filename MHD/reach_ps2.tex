\documentclass{article}
\usepackage[utf8]{inputenc}
\usepackage{graphicx,amsmath,amsfonts,amssymb,mathtools}
\usepackage{enumitem}
\usepackage{empheq}
\usepackage{xcolor}

\newcommand\Ra{\mathrm{Ra}}
\newcommand\Pran{\mathrm{Pr}}
\newcommand\Rac{\mathrm{Ra}_{\mathrm{c}}}
\newcommand\Ek{\mathrm{Ek}}
\newcommand\Ro{\mathrm{Ro}}
\newcommand\Nu{\mathrm{Nu}}
\newcommand\Sc{\mathrm{Sc}}

\newcommand\eps{\varepsilon}
\renewcommand\L {\mathcal{L}}

\newcommand{\n}{\\ \nonumber \\ }
\newcommand{\nn}{\nonumber}
\newcommand{\nnn}{\\ \nonumber \\ \nonumber}

\newcommand\ie{\textit{i.e.},~}
\newcommand\eg{\textit{e.g.},~}
\newcommand{\omicron}{o}
\newcommand*\widefbox[1]{\fbox{\hspace{2em}#1\hspace{2em}}}

\newcommand{\pd}[1]{\partial_{#1}}
\newcommand{\vhat}[1]{\hat{\boldsymbol{#1}}}
\renewcommand{\vec}[1]{\boldsymbol{#1}}
\newcommand{\M}[1]{\mathbf{#1}}
\newcommand{\grad}{\vec{\nabla}}
\newcommand{\cross}{\vec{\times}}
\newcommand{\curl}{\grad \vec{\times}}
\newcommand{\divergence}{\grad \cdot}
\newcommand{\laplacian}{\nabla^2}
\newcommand{\veclaplacian}{\grad^2}

\newcommand{\sump}[2]{\sideset{}{'}\sum_{{#1}=0}^{#2}}

\newcommand{\eq}[1]{(\ref{#1})}
\newcommand{\eqs}[2]{(\ref{#1})~\&~(\ref{#2})}
\newcommand{\eqss}[2]{(\ref{#1})--(\ref{#2})}

\newcommand{\Eq}[1]{Eq.~(\ref{#1})}
\newcommand{\Eqs}[2]{Eqs.~(\ref{#1})~\&~(\ref{#2})}
\newcommand{\Eqss}[2]{Eqs.~(\ref{#1})--(\ref{#2})}

\newcommand{\fig}[1]{Fig.~(\ref{#1})}
\newcommand{\figs}[2]{Figs.~(\ref{#1})~\&~(\ref{#2})}
\newcommand{\T}{{\cal T}}
\newcommand{\Z}{{\cal Z}}

\setlength{\parindent}{0pt}

\begin{document}
\subsection*{Spatial Derivatives: Gradient, Divergence, Laplacian}
\vspace{1cm}
\begin{enumerate}
    \item Consider the scalar function 
    \begin{align*}
        f(x, y) &= \sin\big( \frac{x}{\sqrt{y}} \big) + xy^2 + x.
        \intertext{Notice that $f$ takes a point in 2D space and returns a single number. In math language, this can be written as}
        f\,&:\, \mathbb{R}^2 \to \mathbb{R}.
    \end{align*}
    \newline
    \begin{enumerate}
        \item Compute $f$'s gradient $\grad f$. Remember that the gradient operator acts on a scalar function, and returns a vector field.\newline\newline

        \item Compute $f$'s laplacian $\grad^2 f$. Remember that the laplacian operator acts on a scalar function, and returns a scalar field.\newline\newline

        \item Compute the divergence of $f$'s gradient of $\grad \cdot \grad f$.\newline\newline
    \end{enumerate}

\end{enumerate}

\end{document}
 