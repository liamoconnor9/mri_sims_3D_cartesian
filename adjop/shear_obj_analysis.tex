\documentclass{article}
\usepackage[utf8]{inputenc}
\usepackage{graphicx,amsmath,amsfonts,amssymb,mathtools}
\usepackage{enumitem}
\usepackage{empheq}
\usepackage{xcolor}

\newcommand\Ra{\mathrm{Ra}}
\newcommand\Pran{\mathrm{Pr}}
\newcommand\Rac{\mathrm{Ra}_{\mathrm{c}}}
\newcommand\Ek{\mathrm{Ek}}
\newcommand\Ro{\mathrm{Ro}}
\newcommand\Nu{\mathrm{Nu}}
\newcommand\Sc{\mathrm{Sc}}

\newcommand\eps{\varepsilon}
\renewcommand\L {\mathcal{L}}

\newcommand{\n}{\\ \nonumber \\ }
\newcommand{\nn}{\nonumber}
\newcommand{\nnn}{\\ \nonumber \\ \nonumber}

\newcommand\ie{\textit{i.e.},~}
\newcommand\eg{\textit{e.g.},~}
\newcommand{\omicron}{o}
\newcommand*\widefbox[1]{\fbox{\hspace{2em}#1\hspace{2em}}}

\newcommand{\pd}[1]{\partial_{#1}}
\newcommand{\vhat}[1]{\hat{\boldsymbol{#1}}}
\renewcommand{\vec}[1]{\boldsymbol{#1}}
\newcommand{\M}[1]{\mathbf{#1}}
\newcommand{\grad}{\vec{\nabla}}
\renewcommand{\skew}{\vec{\nabla}^{\perp}}
\newcommand{\cross}{\vec{\times}}
\newcommand{\curl}{\grad \vec{\times}}
\newcommand{\divergence}{\grad \cdot}
\newcommand{\laplacian}{\nabla^2}
\newcommand{\veclaplacian}{\grad^2}

\newcommand{\sump}[2]{\sideset{}{'}\sum_{{#1}=0}^{#2}}

\newcommand{\eq}[1]{(\ref{#1})}
\newcommand{\eqs}[2]{(\ref{#1})~\&~(\ref{#2})}
\newcommand{\eqss}[2]{(\ref{#1})--(\ref{#2})}

\newcommand{\Eq}[1]{Eq.~(\ref{#1})}
\newcommand{\Eqs}[2]{Eqs.~(\ref{#1})~\&~(\ref{#2})}
\newcommand{\Eqss}[2]{Eqs.~(\ref{#1})--(\ref{#2})}

\newcommand{\fig}[1]{Fig.~(\ref{#1})}
\newcommand{\figs}[2]{Figs.~(\ref{#1})~\&~(\ref{#2})}
\newcommand{\T}{{\cal T}}
\newcommand{\Z}{{\cal Z}}
\newcommand{\px}{\partial_x}
\newcommand{\py}{\partial_y}
\newcommand{\pz}{\partial_z}
\setlength{\parindent}{0pt}

\begin{document}

\section*{2D Shear: Analyses of Least-Squares Objective Functions}

\vfill
\subsection*{Velocity Objective}
Given a target simulation's end state $\vec{U}(T)$, consider an objective functional given by

\begin{align*}
    H[\vec{u}(T)] &= (\vec{u}(T) - \vec{U}(T)) \cdot (\vec{u}(T) - \vec{U}(T)) = (\vec{u}(T) - \vec{U}(T))^2
    \intertext{To determine the adjoint problem's initial $(t=T)$ state, we take}
    \frac{\delta H}{\delta \vec{u}(T)} &= \lim_{\delta\vec{u}(T) \to 0} \Big[ \frac{H[\vec{u}(T) + \delta\vec{u}(T)] - H[\vec{u}(T)]}{\delta\vec{u}(T)} \Big] \\
    &= \lim_{\delta\vec{u}(T) \to 0} \Big[ \frac{ (\vec{u}(T) + \delta\vec{u}(T) - \vec{U}(T))^2 - (\vec{u}(T) - \vec{U}(T))^2}{\delta\vec{u}(T)} \Big] \\
    &= \lim_{\delta\vec{u}(T) \to 0} \Big[ \frac{  2\delta\vec{u}(T)(\vec{u}(T) - \vec{U}(T))}{\delta\vec{u}(T)} \Big] \\
    &= 2(\vec{u}(T) - \vec{U}(T))
    \intertext{\textbf{Note:} as we take $T \to 0$, we have that $\vec{u}(0) \to \vec{u}(T)$ and $\vec{U}(0) \to \vec{U}(T)$. From this, it follows that the adjoint system is initialized as}
    \vec{\mu}(T) &= 2(\vec{u}(0) - \vec{U}(0)) \quad \to \quad \vec{\mu}(0) 
    \intertext{Therefore}
    \frac{\delta H}{\delta \vec{u}(0)} &\to 2(\vec{u}(0) - \vec{U}(0))
    \intertext{Which is the shortest direction to the optimized state in $L_2$ space.}
\end{align*}
\vfill

\newpage
\subsection*{Vorticity Objective}
Next, we define
\begin{align*}
    \omega &= \curl \vec{u} \cdot \vec{\hat{k}} \equiv \skew \cdot \vec{u} = \px v - \py u  \\
    W &= \curl \vec{U} \cdot \vec{\hat{k}} \equiv \skew \cdot \vec{U} = \px V - \py U \\
    \intertext{and consider an objective functional given by}
    H[\vec{u}(T)] &= (\omega(T) - W(T))^2 \\
    &= (\px v(T) - \py u(T) - W(T))^2
    \intertext{\textbf{Note:} if we take}
    \frac{\delta H}{\omega(T)} &= 2(\omega(T) - W(T)),
    \intertext{we recover the same result as before. It's more interesting to consider}
    \frac{\delta H}{\delta u(T)} &= \lim_{\delta{u}(T) \to 0} \Big[ \frac{(\px v(T) - \py u(T) - \py \delta{u}(T) - W(T))^2 - (\px v(T) - \py u(T) - W(T))^2}{\delta{u}(T)} \Big] \\
    &= \lim_{\delta{u}(T) \to 0} \Big[ \frac{ - \py \delta{u}(T) \big(\px v(T) - \py u(T) - W(T)\big)}{\delta{u}(T)} \Big] \\
    \intertext{integrating by parts gives}
    &= \lim_{\delta{u}(T) \to 0} \Big[ \frac{ - \py \Big[\delta{u}(T) \big(\px v(T) - \py u(T) - W(T)\big) \Big]  + \delta{u}(T)\py \Big[ \px v(T) - \py u(T) - W(T) \Big]}{\delta{u}(T)} \Big] \\
    \intertext{The first term in the numerator cancels when imprenetrable boundary conditions are employed, leaving}
    &= \lim_{\delta{u}(T) \to 0} \Big[ \frac{ \delta{u}(T)\py \Big[ \px v(T) - \py u(T) - W(T) \Big]}{\delta{u}(T)} \Big] \\[0.4cm]
    &= \py \Big[ \omega(T) - W(T) \Big] \\
    \intertext{Applying the same process to $\frac{\delta H}{\delta v(T)}$, we find }
    \frac{\delta H}{\delta \vec{u}(T)} &=  \skew \Big[ \omega(T) - W(T) \Big] \\
\end{align*}

\newpage
\subsection*{Streamfunction objective}
We define the streamfunctions $\psi(x, y)$ and $\phi(x, y)$ satisfying
\begin{align*}
    \skew \psi &= \vec{u} \\ 
    \skew \phi &= \vec{U} \\ 
    \intertext{Note that}
    \skew\cdot\skew\psi &= \laplacian \psi = \skew \cdot \vec{u} = \omega\\
    \skew\cdot\skew\phi &= \laplacian \phi = \skew \cdot \vec{U} = W\\
    \intertext{Consider the original velocity objective}
    H[\vec{u}(T)] &= (\vec{u}(T) - \vec{U}(T)) \cdot (\vec{u}(T) - \vec{U}(T)) \\
    &= (\vec{u}(T) - \vec{U}(T))^2 \\
    &= (\skew\psi(T) - \skew\phi(T))^2 \\
    \intertext{We then take}
    \frac{\delta H}{\delta \psi(T)} &= \lim_{\delta\psi(T) \to 0} \frac{(\skew(\psi(T)+\delta\psi(T)) - \skew\phi(T))^2 - (\skew\psi(T) - \skew\phi(T))^2}{\delta\psi(T)}\\
     &= \lim_{\delta\psi(T) \to 0} \frac{ 2\delta\psi(T)(\skew\psi(T) - \skew\phi(T))}{\delta\psi(T)}\\
     &= 2(\skew\psi(T) - \skew\phi(T))\\
     &= 2(\vec{u}(T) - \vec{U}(T))
    \intertext{This is the same thing as before! To get a new adjoint ic, we need to take}
    H[\psi(T)] &= (\psi(T) - \phi(T))^2
    \intertext{along with}
    \frac{\delta H}{\delta \vec{u}(T)} &= \frac{\delta}{\delta \vec{u}(T)} [(\psi(T) - \phi(T))^2]
    \intertext{Thus, by applying the appropriate boundary conditions, we can solve for $\psi$ given a flow field.}
    \psi &= \grad^{-2} \omega \\
    \phi &= \grad^{-2} W \\
    \intertext{Recall that we are interested in a streamfunction objective's variational derivative with respect to velocity, as differentiating the least-squares functional $(\psi - \phi)^2$ wrt $\psi$ gives us the same thing as before!}
    H[\psi(T)] &= (\psi - \phi)^2 \\
    &= (\grad^{-2}\omega - \grad^{-2}W)^2 \\
    &= (\grad^{-2}\omega)^2 - 2\grad^{-2}\omega\grad^{-2}W + (\grad^{-2}W)^2 \\
    \intertext{Taking the variation}
    \frac{\delta H}{\delta u} &= \lim_{\delta u \to 0} \frac{2\grad^{-2}\skew\cdot\vec{u}\grad^{-2}\skew\cdot\delta\vec{u} - 2\grad^{-2}\skew\cdot\delta\vec{u}\grad^{-2}W}{\delta u} \\
     &= \lim_{\delta u \to 0} \frac{2\grad^{-2}\skew\cdot\delta\vec{u}(\grad^{-2}\skew\cdot\vec{u} - \grad^{-2}W)}{\delta u} \\
     &= 2(\psi - \phi)\lim_{\delta u \to 0} \frac{\grad^{-2}\skew\cdot\delta\vec{u}}{\delta u} \\
\end{align*}

\end{document}
 