\documentclass{article}
\usepackage[utf8]{inputenc}
\usepackage{graphicx,amsmath,amsfonts,amssymb,mathtools}
\usepackage{enumitem}
\usepackage{empheq}
\usepackage{xcolor}

\newcommand\Ra{\mathrm{Ra}}
\newcommand\Pran{\mathrm{Pr}}
\newcommand\Rac{\mathrm{Ra}_{\mathrm{c}}}
\newcommand\Ek{\mathrm{Ek}}
\newcommand\Ro{\mathrm{Ro}}
\newcommand\Nu{\mathrm{Nu}}
\newcommand\Sc{\mathrm{Sc}}

\newcommand\eps{\varepsilon}
\renewcommand\L {\mathcal{L}}

\newcommand{\n}{\\ \nonumber \\ }
\newcommand{\nn}{\nonumber}
\newcommand{\nnn}{\\ \nonumber \\ \nonumber}

\newcommand\ie{\textit{i.e.},~}
\newcommand\eg{\textit{e.g.},~}
\newcommand{\omicron}{o}
\newcommand*\widefbox[1]{\fbox{\hspace{2em}#1\hspace{2em}}}

\newcommand{\pd}[1]{\partial_{#1}}
\newcommand{\vhat}[1]{\hat{\boldsymbol{#1}}}
\renewcommand{\vec}[1]{\boldsymbol{#1}}
\newcommand{\M}[1]{\mathbf{#1}}
\newcommand{\grad}{\vec{\nabla}}
\renewcommand{\skew}{\vec{\nabla}^{\perp}}
\newcommand{\cross}{\vec{\times}}
\newcommand{\curl}{\grad \vec{\times}}
\newcommand{\divergence}{\grad \cdot}
\newcommand{\laplacian}{\nabla^2}
\newcommand{\veclaplacian}{\grad^2}

\newcommand{\sump}[2]{\sideset{}{'}\sum_{{#1}=0}^{#2}}

\newcommand{\eq}[1]{(\ref{#1})}
\newcommand{\eqs}[2]{(\ref{#1})~\&~(\ref{#2})}
\newcommand{\eqss}[2]{(\ref{#1})--(\ref{#2})}

\newcommand{\Eq}[1]{Eq.~(\ref{#1})}
\newcommand{\Eqs}[2]{Eqs.~(\ref{#1})~\&~(\ref{#2})}
\newcommand{\Eqss}[2]{Eqs.~(\ref{#1})--(\ref{#2})}

\newcommand{\fig}[1]{Fig.~(\ref{#1})}
\newcommand{\figs}[2]{Figs.~(\ref{#1})~\&~(\ref{#2})}
\newcommand{\T}{{\cal T}}
\newcommand{\Z}{{\cal Z}}
\newcommand{\px}{\partial_x}
\newcommand{\py}{\partial_y}
\newcommand{\pz}{\partial_z}
\setlength{\parindent}{0pt}

\begin{document}

\section*{1D: Diffusion Inversion}
\subsection*{Preliminaries}
\begin{align*}
    \intertext{We begin by defining the Lagrangian}
    \L &= \int_0^T \langle \mu \cdot [\partial_t u - \nu \partial^2_x u] \rangle dt + \frac{1}{2}\langle (u(T) - U(T))^2 \rangle
    \intertext{where $U(T)$ is the target end state, whose initial condition $U(0)$ we wish to recover. We continue by deriving the forward and adjoint problems}
    0 = \frac{\delta\L}{\delta\mu} &= \int_0^T \langle \partial_t u - \nu \partial^2_x u \rangle dt \quad \longrightarrow \quad \partial_t u = \nu \partial^2_x u 
    \intertext{Before deriving the adjoint system, we note that}
    \L &= \int_0^T \langle \mu \partial_t u - \nu \mu\partial^2_x u \rangle dt + \frac{1}{2}\langle (u(T) - U(T))^2 \rangle\\
    &= \int_0^T \langle  \partial_t (u\mu) - u\partial_t \mu - \nu \mu\partial^2_x u \rangle dt + \frac{1}{2}\langle (u(T) - U(T))^2 \rangle\\
    &= \Big\langle\int_0^T   \partial_t (u\mu) dt\Big\rangle + \int_0^T \langle- u\partial_t \mu - \nu \mu\partial^2_x u \rangle dt + \frac{1}{2}\langle (u(T) - U(T))^2 \rangle\\
    &= \langle   u(T)\mu(T) - u(0)\mu(0) \rangle + \int_0^T \langle- u\partial_t \mu - \nu \mu\partial^2_x u \rangle dt + \frac{1}{2}\langle (u(T) - U(T))^2 \rangle\\
    \intertext{We continue by deriving the adjoint system. For periodic (among other) boundary conditions}
    0 = \frac{\delta\L}{\delta u} &= -\int_0^T \langle \partial_t\mu + \nu\partial^2_x \mu \rangle dt \quad \longrightarrow \quad \partial_t \mu = -\nu\partial^2_x \mu
    \intertext{To initialize the adjoint system at $t=T$, we require}
    0 = \frac{\delta\L}{\delta u(T)} &= \langle \mu(T) + u(T) - U(T) \rangle \quad \longrightarrow \quad \mu(T) = U(T) - u(T)
    \intertext{To determine a trajectory which is coincident with $U(0)$, we take}
    \frac{\delta\L}{\delta u(0)} &= -\mu(0) 
\end{align*}

\newpage
\subsection*{Linearity Cancellation}
\begin{align*}
    \intertext{Here we demonstrate that the target state $U(0)$ manifests itself as a constant shift in the objective function wrt $u(0)$. Let $\tilde{f}$ represent a function $f$ which has undergone diffusion with diffusivity $\nu$ over a time period $T$, i.e.}
    u(0) &= f \quad \longrightarrow \quad u(T) = \tilde{f}
    \intertext{Suppose our guess's deviation from the target initial state is given by $-w$, i.e.}
    u(0) &= U(0) - w
    \intertext{Then}
    u(T) &= U(T) - \tilde{w} \quad \longrightarrow \quad \frac{1}{2}\langle (u(T) - U(T))^2\rangle = \frac{1}{2}\langle w^2 \rangle
    \intertext{which is independent of the target state. So the objective only depends on our deviation from the target state, rather than the target state itself.}
\end{align*}

\subsection*{Objective Level Sets}
\begin{align*}
    \intertext{Given that the target state doesn't matter for the following analysis, we take $U(0) = U(T) = 0$ for all $x$}
    \intertext{Suppose}
    u(0) &= a_1 e^{i k_1 x} + a_2 e^{i k_2 x} + ...
    \intertext{We wish to characterize the objective's dependence on the initial guess in this $n$-dimensional phase space $\vec{a} = (a_1, a_2, ..., a_n) \in \mathbb{C}^n$. We can solve for $u(T)$ analytically}
    u(T) &= a_1 e^{i k_1 x - \nu k_1^2 T} + a_2 e^{i k_2 x - \nu k_2^2 T} + ...\\
    &= a_1 e^{- \nu k_1^2 T}e^{i k_1 x } + a_2 e^{- \nu k_2^2 T}e^{i k_2 x } + ...
    \intertext{The objective }
    \frac{1}{2}\langle |u(T)|^2 \rangle &= \frac{1}{2}\langle \big( a_1 e^{- \nu k_1^2 T}e^{i k_1 x } + a_2 e^{- \nu k_2^2 T}e^{i k_2 x } + ...\big)^2 \rangle \\
    &= \frac{1}{2}\Big[a_1^2 e^{- 2\nu k_1^2 T} \langle \cos^2(k_1 x) + \sin^2(k_1 x) \rangle + a_2^2 e^{- 2\nu k_2^2 T} \langle \cos^2(k_2 x) + \sin^2(k_2 x)  \rangle  + ... \Big]\\
    &= \big(\pi e^{- 2\nu k_1^2 T}\big)\; a_1^2 + \big(\pi e^{- 2\nu k_2^2 T}\big) \; a_2^2 + ... + \big(\pi e^{- 2\nu k_n^2 T}\big) \; a_n^2  \\
    \intertext{Therefore the objective's level sets are given by infinite-dimensional hyperellipses in $\vec{a}$ space.}
\end{align*}

\subsection*{Preconditioner}
\begin{align*}
    \intertext{The appropriate preconditioner should be obvious: we need to undiffuse the adjoint initial condition twice! Doing so remaps the levelsets of hyperellipses into hyperspheres, thereby ensuring that the gradient is always in the direction of $u(0) - U(0)$}
    \intertext{Suppose the fourier coefficients of the adjoint initial condition are given by $\vec{b} = (b_1, b_2, b_3, ..., b_n) \in \mathbb{C}^n$. Then it would be advantageous to initialize the adjoint system with fourier coefficients $\tilde{\vec{b}}$ where}
    \begin{bmatrix}
        \tilde{b_1}\\
        \tilde{b_2}\\
        \tilde{b_3}\\
        \vdots \\
        \tilde{b_n}\\
    \end{bmatrix}
    &= 
    \begin{bmatrix}
        e^{2\nu k_1^2T} & 0 & 0 & \hdots & 0\\
        0 & e^{2\nu k_2^2T} & 0 & \hdots & 0\\
        0 & 0 & e^{2\nu k_3^2T} & \hdots & 0\\
        \vdots & \vdots & \vdots & \ddots & \vdots \\
        0 & 0 & 0 & \hdots & e^{2\nu k_n^2T}\\
    \end{bmatrix}
    \begin{bmatrix}
        {b_1}\\
        {b_2}\\
        {b_3}\\
        \vdots \\
        {b_n}\\
    \end{bmatrix}
\end{align*}

\end{document}
 