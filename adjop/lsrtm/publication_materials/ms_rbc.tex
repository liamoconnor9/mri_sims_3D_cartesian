\documentclass[longbibliography,twocolumn,amsmath,amssymb,aps,nofootinbib]{revtex4-2}

\usepackage{graphicx}
\usepackage{natbib}
\usepackage{amsmath}
\usepackage{longtable}
\usepackage{gensymb}
\usepackage{enumerate}
\usepackage{varwidth}
\usepackage{float}
\usepackage{nonfloat}
\usepackage{soul}
\usepackage{ulem}
% \usepackage{multicol}
% \usepackage{afterpage}
\usepackage[usenames, dvipsnames]{color}
\newcommand{\note}[1]{\noindent \textbf{\textit{\textcolor{Red}{#1}}}}

\newcommand\Ra{\mathrm{Ra}}
\newcommand\Pran{\mathrm{Pr}}
\newcommand\Rac{\mathrm{Ra}_{\mathrm{c}}}
\newcommand\Ek{\mathrm{Ek}}
\newcommand\Ro{\mathrm{Ro}}
\newcommand\Nu{\mathrm{Nu}}
\newcommand\Sc{\mathrm{Sc}}

\newcommand\eps{\varepsilon}
\renewcommand\L {\mathcal{L}}
\renewcommand{\citet}[1]{ref.~\cite{#1}}
\renewcommand{\Citet}[1]{Ref.~\cite{#1}}
\newcommand{\citets}[1]{refs.~\cite{#1}}
\newcommand{\Citets}[1]{Refs.~\cite{#1}}

\newcommand{\n}{\\ \nonumber \\ }
\newcommand{\nn}{\nonumber}
\newcommand{\nnn}{\\ \nonumber \\ \nonumber}

\newcommand\ie{\textit{i.e.},~}
\newcommand\eg{\textit{e.g.},~}
\newcommand{\omicron}{o}

\newcommand{\pd}[1]{\partial_{#1}}
\renewcommand{\vec}[1]{\boldsymbol{#1}}
\newcommand{\M}[1]{\mathbf{#1}}
\newcommand{\grad}{\vec{\nabla}}
\newcommand{\cross}{\vec{\times}}
\newcommand{\laplacian}{\nabla^2}

\newcommand{\sump}[2]{\sideset{}{'}\sum_{{#1}=0}^{#2}}

\newcommand{\eq}[1]{(\ref{#1})}
\newcommand{\eqs}[2]{(\ref{#1})~\&~(\ref{#2})}
\newcommand{\eqss}[2]{(\ref{#1})--(\ref{#2})}

\newcommand{\Eq}[1]{Eq.~(\ref{#1})}
\newcommand{\Eqs}[2]{Eqs.~(\ref{#1})~\&~(\ref{#2})}
\newcommand{\Eqss}[2]{Eqs.~(\ref{#1})--(\ref{#2})}

\newcommand{\fig}[1]{Fig.~(\ref{#1})}
\newcommand{\figs}[2]{Figs.~(\ref{#1})~\&~(\ref{#2})}
\newcommand{\T}{{\cal T}}
\newcommand{\Z}{{\cal Z}}


\makeatletter
\let\Hy@backout\@gobble
\makeatother

\newcommand*{\GtrSim}{\smallrel\gtrsim}

\makeatletter
\newcommand*{\smallrel}[2][.8]{%
  \mathrel{\mathpalette{\smallrel@{#1}}{#2}}%
}
\newcommand*{\smallrel@}[3]{%
  % #1: scale factor
  % #2: math style
  % #3: symbol
  \sbox0{$#2\vcenter{}$}%
  \dimen@=\ht0 %
  \raise\dimen@\hbox{%
    \scalebox{#1}{%
      \raise-\dimen@\hbox{$#2#3\m@th$}%
    }%
  }%
}
\makeatother


\begin{document}

\title{Marginally-Stable Thermal Equilibria of Rayleigh-B\'enard Convection}

\author{Liam O'Connor$^{1,2}$}
\affiliation{%
$^1$Department of Engineering Sciences and Applied Mathematics, Northwestern University, Evanston, IL 60208 USA}
\affiliation{%
$^2$Center for Interdisciplinary Exploration and Research in Astrophysics, Northwestern University, Evanston, IL, 60201 USA}

\begin{abstract}
    Natural convection exhibits turbulent flows which are difficult or impossible to resolve in direct numerical simulations. In this work, we investigate a quasilinear form of the Rayleigh-B\'enard problem which describes the bulk one-dimensional properties of convection without resolving the turbulent dynamics. We represent perturbations away from the mean using a sum of marginally-stable eigenmodes.  By constraining the perturbation amplitudes, the marginal stability criterion allows us to evolve the background temperature profile under the influence of multiple eigenmodes representing flows at different length scales.
    We find the quasilinear system evolves to an equilibrium state where advective and diffusive fluxes sum to a constant.
    These marginally-stable thermal equilibria (MSTE) are exact solutions of the quasilinear equations.
    The mean MSTE temperature profiles have thinner boundary layers and larger Nusselt numbers than thermally-equilibrated 2D and 3D simulations of the full nonlinear equations.  
    MSTE solutions exhibit a classic boundary-layer scaling of the Nusselt number $\Nu$ with the Rayleigh number $\Ra$ of $\Nu \sim \Ra^{1/3}$.
    When an MSTE is used as initial conditions for a 2D simulation, we find that Nu quickly equilibrates without the burst of turbulence often induced by purely conductive initial conditions, but we also find that the kinetic energy is too large and viscously attenuates on a long viscous time scale.
\end{abstract}

\maketitle

\section{Introduction}
Rayleigh-B\'enard convection plays a foundational role in astrophysical and geophysical settings.
The resulting buoyancy-driven flows regulate heat transfer and generate large-scale vortices \cite{Couston}.
Turbulent convection, which is associated with large Rayleigh numbers $\Ra$, is difficult to simulate. 
State of the art simulations performed by \cite{Zhu_2018} have reached $\Ra \sim 10^{14}$ but estimates for the sun's convective zone and earth's interior are $\Ra \sim 10^{16}-10^{20}$ and $\Ra \sim 10^{20}-10^{30}$ respectively \cite{Ossendrijver,Gubbins_2001}. 


\bibliography{ms_rbc.bib}

\end{document}
